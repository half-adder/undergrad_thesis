\chapter{Discussion}

\label{Chapter4}

Genetically encoded redox-sensitive fluorescent biomarkers have spurred a revolution in the study of dynamic cellular redox processes. While the ratiometric nature of these sensors yields many desirable qualities, it also necessitates careful analysis in the event of inter-frame movement. 

This thesis has introduced how inter-frame movement introduces error in ratiometric fluorescent microscopy and proposed three strategies to mitigate this error in the context of the pharyngeal muscle of the nematode \textit{C. elegans}. Frame-specific masks allow for flexible measurement boundaries, which achieves a limited form of global linear registration. Frame-specific midlines account for the dorsoventral movement common at the anterior regions of the pharynx. A functional approach to registration accounts for the differences in arc-length of the frame-specific midlines, and achieves the non-linear registration that frame-specific midlines require. These improvements allow previously unusable data to be re-analyzed and will require fewer animals to be excluded in future experiments.

This thesis has also shown how new approaches to segmentation and centerline estimation reduces the need for manual input in the pipeline via incorporation of edge information and transmitted-light information, respectively. In most cases, the pipeline requires no manual input and analysis can be done with little effort and time, increasing experimental throughput.

%-------------------------------------------------------------------------------------
%	SECTION 1
%-------------------------------------------------------------------------------------
\section{Future Directions}

% SUBSECTION
\subsection{Automatic movement detection}

This thesis has focused on the mitigation of the error introduced by inter-frame movement. As noted, not all error can be removed by these methods. As such, it would be useful for an automated classification system to detect inter-frame movement. The fundamental difficulty with this task is that in an experimental setting the image pairs consist of one image at 410 nm and another at 470 nm. Because the excitation spectra of roGFP is different at each wavelength, it is difficult to determine if the differences in each image's intensity profile is due to movement or true biochemical activity. This concern could be addressed by quantifying distance between midlines instead of difference in intensity profile. Alternatively, different imaging strategies could be developed. For example, if four images were taken of each animal at alternating excitation wavelengths, differences in the pairs with the same wavelength could be used to quantify error and thereby inter-frame movement.

% SUBSECTION
\subsection{Extension to different tissues}

The analysis pipeline described in this thesis is highly specific to the pharyngeal muscle. Due to its stereotyped geometry, this tissue has been an ideal foundation on which to build. However, there is considerable interest in understanding dynamic redox processes in other tissues such as the intestine and nervous system. Each tissue will surely introduce its own challenges that will need to be individually addressed. Precise methodologies for quantifying redox processes in these different tissues will lead to a deeper understanding of how these processes are regulated at an organismal level.

%-------------------------------------------------------------------------------------
%	SECTION 2
%-------------------------------------------------------------------------------------
