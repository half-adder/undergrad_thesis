% Chapter Template

\chapter{Introduction} % Main chapter title

\label{Chapter1} % Change X to a consecutive number; for referencing this chapter elsewhere, use \ref{ChapterX}

%----------------------------------------------------------------------------------------
%	SECTION 1
%----------------------------------------------------------------------------------------

\section{Biology}

%-----------------------------------
%	SUBSECTION 1
%-----------------------------------
\subsection{ROS Play a Central Role in Biology}

ROS are a highly reactive class of molecules, and interact with a wide array of macromolecules inside the cell including DNA, lipids, and proteins. The bases and the backbone of DNA are both damaged by ROS attack. The polyunsaturated fatty acid residues of phospholipids are extremely sensitive to oxidation by ROS. The side chains of all amino acids are vulnerable to oxidation by ROS4. Cysteine and methionine residues are especially sensitive to oxidation, forming reversible disulfide bridges between thiol groups.
ROS do not only function deleteriously. They are also believed to function crucially in many cellular signaling pathways. Changes in the thiol status of proteins due to changes in the redox environment of the cell can have wide and disparate downstream effects. H2O2, for example, is believed to act as a secondary messenger. It reacts specifically with cysteine residues to create a cysteinesulfenic acid, which can react with another cysteine (leading to conformational changes), GSH (producing a glutathionylated protein), amides, and hydroperoxides5. Glutathionylation is essential to the function of PTP1B, a negative regulator of the insulin signaling pathway in humans, the misregulation of which is implicated in type II diabetes and cancer. Indeed, redox state has been found to mediate a great many signaling pathways. Thioredoxin acts as a ROS sensor inhibiting apoptosis. Redox state also affects to the differential regulation of transcriptional factors/activators including p53, AP-1, and NF-kB10. Thus, redox balance affects not only transient protein chemistry, but the differential expression of thousands of downstream targets of transcription networks.

%-----------------------------------
%	SUBSECTION 2
%-----------------------------------
\subsection{roGFP Enables Quantification of Redox State}

The redox environment of the cell is comprised of many linked redox couples. Each couple is a pair of molecules which can be reversibly oxidized and reduced and thus exists in one of two states inside the cell. Glutathione (GSH) for example can act as an electron donor to become its oxidized form glutathione disulfide (GSSG). It may then be reduced back to GSH with NADPH as the electron donor. We calculate the half-cell reduction potential of that couple via the Nernst equation and call this the redox state of the couple. There are many redox couples inside the cell. The GSSG/2GSH couple is most abundant and the reduction potential of this couple seems to change with biologically relevant phenomena (proliferation, differentiation, apoptosis). It is considered to be the major contributor to redox environment, and so many researchers use the redox state of this couple as a proxy for the redox state inside the cytosol as a whole.

%----------------------------------------------------------------------------------------
%	SECTION 2
%----------------------------------------------------------------------------------------

\section{Pipline Background}

Sed ullamcorper quam eu nisl interdum at interdum enim egestas. Aliquam placerat justo sed lectus lobortis ut porta nisl porttitor. Vestibulum mi dolor, lacinia molestie gravida at, tempus vitae ligula. Donec eget quam sapien, in viverra eros. Donec pellentesque justo a massa fringilla non vestibulum metus vestibulum. Vestibulum in orci quis felis tempor lacinia. Vivamus ornare ultrices facilisis. Ut hendrerit volutpat vulputate. Morbi condimentum venenatis augue, id porta ipsum vulputate in. Curabitur luctus tempus justo. Vestibulum risus lectus, adipiscing nec condimentum quis, condimentum nec nisl. Aliquam dictum sagittis velit sed iaculis. Morbi tristique augue sit amet nulla pulvinar id facilisis ligula mollis. Nam elit libero, tincidunt ut aliquam at, molestie in quam. Aenean rhoncus vehicula hendrerit.